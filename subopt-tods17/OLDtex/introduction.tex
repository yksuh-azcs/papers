\section{Introduction}
\todo{Rick:motivate suboptimality. } 
\shorten{Put back flutter for journal version of paper}
DBMSes
underlie all information systems and hence optimizing their performance is of
critical importance. One could argue that optimizing a query's performance
could be done by the software engineer. However, the time taken to examine,
identify and correct inefficiencies for every query in the DBMS would result
in a huge expenditure of time and effort. What needs to be done instead is
to systematically examine the factors influencing the number of suboptimal
queries. There could be multiple causes of the suboptimality: One possible
cause could be some peculiarity within the tens of thousands of lines of
code of that query optimizer. Another possible cause could be the query
complexity. A third possible reason could be some fundamental limitation
{\em within the general architecture of cost-based optimization} that will
always render a good number of queries suboptimal. Prior research shows that
increasing the complexity negatively influences
performance~\cite{campbell88,moody98}.

To better understand the impact of different factors on suboptimality of
query performance and the interaction between operators, especially in a
dynamic environment, an experimental approach is needed. This is where our
research makes its contribution.  Our research introduces a novel approach
to better understand the factors influencing query performance in
DBMSes. Based on existing research and general knowledge of DBMSes, we
developed a simple predictive model to predict performance, in particular
suboptimality of query performance. We use an experimental methodology with
the DBMS as a subject to test our hypotheses with respect to factors
influencing suboptimality. Our research falls within creative development of
new evaluation methods and metrics for artifacts, which were identified as
important design-science contributions~\cite{hevner04}.

The key contributions of our research are as follows.
\begin{itemize}
\item It builds and tests a predictive model for DBMSes to better understand the factors influencing performance.

\item It uses an innovative methodology that treats DBMSes as experimental subjects.

\item The predictive model has some unsettling practical implications.
\end{itemize}

This paper takes a scientifically rigorous approach to an
area previously dominated by the engineering perspective, that of database
query optimization.  Our goal is to understand cost-based query optimizers
as a {\em general} class of computational artifacts and to come up with
insights and ultimately with predictive theories about how such optimizers,
again, as a general class, behave.  These theories can be used to further
improve DBMSes through engineering efforts that benefit from the fundamental
understanding that the scientific perspective can provide.

We focus here on one aspect, that of the effectiveness of query
optimization. The query optimization phase within a DBMS ostensibly finds the fastest
query execution plan from a potentially large set of enumerated plans, all of
which correctly compute the specified query. Occasionally the ``optimizer''
selects the wrong plan, for a variety of reasons. We define {\em
  suboptimality} as the existence of a query plan that performs more
efficiently than the plan chosen by the query optimizer. From the engineering
perspective, it is of critical importance to understand the prevalence
of suboptimality as well as how it arises.

We quantitatively study these aspects \c2j{}{across DBMSes }to identify the
underlying causes. We have developed an initial theory that identifies two
factors that may play a role in suboptimality. The ultimate goal is to {\em
  understand} a component of a DBMS, its cost-based optimizer, through the
articulation and empirical testing of a general scientific theory.

In Section~\ref{sec:related} we briefly summarize the vast amount of related
work in query optimization to establish the technical basis for our study.
The following section introduces the methodology we will follow, that of
empirical generalization. We then present a predictive, causal model of
suboptimality and empirically test several hypotheses derived from that
model\c2j{. These are the first scientific results that we are aware of that
  test a causal model of DBMSes in general.}{ on three DBMSes.
  These are the first scientific results that we are aware of that apply
  {\em across} DBMSes, rather than on a single, specific DBMS or on a
  specific algorithm.}
