\documentclass{sig-alternate}
%\documentclass{vldb}
% \usepackage{fullpage}
% \usepackage{times}
\usepackage{url}
\usepackage{algorithm}
\usepackage{algorithmic}
\usepackage{epsfig}
% \usepackage[total={6.5in,9in}, top=1.25in,left=0.9in]{geometry}
%\usepackage[text={6.5in,9in},centerpage,includefoot]{geometry}
% \usepackage{setspace}
\usepackage{color}
\usepackage{url}
\usepackage{balance}

\definecolor{grey}{RGB}{200,200,200}
\newcommand{\hilite}[1]{\colorbox{grey}{#1}}
\newcommand{\hilitey}[1]{\colorbox{yellow}{#1}}
\newcommand{\hiliting}[1]{\colorbox{grey}{#1}}
\long\def\todo#1{\hilitey{{\bf TODO:} {\em #1}}}
\long\def\shorten#1{}

%For last edit: revise and resubmit for SIGMOD'14
\long\def\shortenF#1#2{}%First is letter indicating what has been removed

%For double blind, keep the second argument. To turn off doubleblind, keep
%the first
\long\def\double#1#2{#2}
\def\azdb{\double{\hbox{\sc AZDBLab}}{\hbox{\sc DBLab}}}
\def\QatC{Q{@}C}
\long\def\comment#1{}
%c2j: Conference to Journal: first parameter is conference, second is journal
% For Conferences: c2j#1#2{#1}
% For Journals: c2j#1#2{2}
\long\def\c2j#1#2{#1}

\begin{document}

{\centering\Large\bf Comments to the Reviewers}

We thank the reviewers for their very clear statement of the concerns they
had, enabling us to work over the last few weeks to articulate better what is
surprising and what is actionable.

\vspace{2ex}\noindent{\bf Meta-review}

\begin{itemize}
\item ``the reviewers felt that the derived conclusions are (1) non
  surprising, i.e., Well known, and (2) non actionable, i.e., Nothing one
  can do about it.''

The last three pages (Sections 7--9) have been substantially rewritten to list explicitly the
surprising results and to indicate the specific things that can (as well as
one that need not) be done, deriving from the results of this research. To
make space for this longer discussion, we removed about 1/2 page of
methodological details that are less important.

We know of no other paper that has presented those results or has been able
to provide the rather specific guidance across DBMSes to database
researchers and developers of places to improve and places where the problem
is not present.

\item ``Furthermore, in the discussion the relationship to Haritsas work was
  pointed out.''
We now explain in Section 7 how our results go beyond Haritsa's novel work. (We
mention here but not in the paper that Haritsa et al. did not look at all at
query suboptimality.)
\end{itemize}

In the following, we respond to just the high-level concerns expressed in
the detailed reviews, though emphasize that we have made small changes
throughout the paper in response to this feedback.

\vspace{2ex}\noindent{\bf Reviewer 1}
\begin{itemize}
\item ``...the results don't specify how the results can be used to improve
  the space of solutions in query optimization.''

Our paper is fundamentally about articulating and testing a causal model
that will thus provide novel guidance to us and to the rest of the community
about how to improve query optimization. This model identifies more
precisely than before where more work could substantially add to the space
of solutions in query optimization.

\item `` There isn't any discussion on
  how the results be integrated to improve the present DBMSes or tune the
  schema or queries better.''

We now discuss in some detail which research questions need more attention
and which do not. We feel that the ultimate goal is to have the DBMS handle
query optimization, and so do not consider manual schema or query tuning.

\item "complex sub-queries ... a discussion on how the model might extend to
  such cases is missing.''

There are several paragraphs in Section 9, modified and expanded, which discuss extension
of the model.

\end{itemize}

\newpage%\vspace{2ex}
\noindent{\bf Reviewer 2}
\begin{itemize}
\item ``There is nothing surprising in the findings and although the
  methodology is interesting, it does not seem to lead to any new insight.''

See the discussion above for the changes we have made.

\item ``It is a bit annoying to read these conclusions without any reference
  to the literature where such problems are addressed.''

We have added some additional references.

What our research does is help focus attention on more specifically where
the problems still reside in query optimization. For example, even when the
statistics are entirely accurate, existing query optimizers have challenges
even on simple queries, a finding that this paper contributes.

\item ``since no attempt is made to put the results in perspective over what
  has been done in the past''

We have not made such an attempt because we don't know (can't know) which
research results from the past have been adequately incorporated  into
existing proprietary DBMSes. Our audience comprises both researchers, who
can develop new solutions, and developers, who can incorporate existing and
new solutions into subsequent versions of their products.

\item ``this could be as simple as mentioning how these problems have been
  addressed in the past and saying that the paper now qualitatively confirms
  what was informally known before.''

This is an excellent suggestion. We mention that in the fourth paragraph of Section
4 and the third bullet of Section 8. We also repeatedly discuss where {\em
  more} research is needed, so as to not imply that research hasn't been
done in the past.

\item ``Put the result in context to what is known already and state if
  there are any conclusions that are different from the state-of-the-art in
  terms of query plans and optimizations.''

We have tried to be as clear and directed as we can, given the constraints
mentioned above.
\end{itemize}

\vspace{2ex}\noindent{\bf Reviewer 3}
\begin{itemize}
\item ``W1: It is not clear what is actionable about the hypotheses.''

See above about what we have added.

\item ``Sure, number of ``effective plans'', number of joins, etc will
  result in more complex plans, and more points where things are suboptimal,
  but most of these are not factors that a query optimizer can control.''\

No disagreement here. These factors do have implications on challenges that
query optimizers much contend with.

\newpage
\item ``In direct contrast, reasons such as errors in statistics,
  independence assumptions, which can be ameliorated by better/more
  expensive statistics are not covered at all.''

We discussed that in our investigation, the statistics are accurate and the
independence assumptions are also accurate (there is not skew in our data),
and so those are not sources for the query suboptimality observed in our
study. Sections 6.7 and 7 discuss the challenges to be addressed and possible
engineering approaches that might help.

\item ``W2: $\ldots$''

This paragraph was removed.

\item ``W3: The paper oversells in parts, eg the claim about manual
  optimization of queries in the introduction is meaningless since most
  queries are canned queries which are executed repeatedly.''

In our experience, canned queries are often those that are most problematic
and require the most manual tuning. In any case, after clarifying that
paragraph, we removed it due to space constraints.

\item ``W4: The paper only looks at one manifestation of suboptimality,
  which respect to plans that are optimal for nearby regions, i.e., only
  checking if the cutover point is chosen properly.''

Actually, we are not considering the specific case that the cutover point is
chosen properly or improperly. Rather, change points are used in our 
operationalization of suboptimality, especially for proprietary systems. So
change points are a fundamental notion in our methodology.

But the reviewer is correct that our definition of suboptimality is
conservative, in that we might miss some suboptimal plans. We mention that
near the end of Section 5.3.

\item ``W5: the experiments are not on any standard benchmark.''

We don't see how that invalidates any of the findings or conclusions we
report.

\item ``Why have you not explained what are the queries you used.''

The queries are generated randomly using the process detailed in Section 5.2.
\end{itemize}

\vspace{2ex}\noindent{\bf Postscript}

Our sincere thanks for the clear feedback, which has substantially benefited
the paper.
\end{document}

% LocalWords:  PVLDB
