\documentclass[10pt,letterpaper]{article}

\usepackage{graphicx}
\usepackage{longtable}
\usepackage{algorithmic}
\usepackage{alltt}
\usepackage{array}
\usepackage[cmex10]{amsmath}
\usepackage{amssymb}
%\usepackage[caption=false]{caption}
\usepackage[dvipsnames]{color}
%\usepackage{colortbl}
%\usepackage{enumitem}
%\usepackage{eqparbox}
%\usepackage{fixltx2e}
%\usepackage{float}
%\usepackage{floatflt}
%\usepackage{mdwmath}
%\usepackage{mdwtab}
%\usepackage{multirow}
\usepackage{stfloats}
%\usepackage[font=footnotesize]{subfig}
\usepackage[caption=false,font=normalsize,labelfont=sf,textfont=sf]{subfig}
%\usepackage[tight,normalsize,sf,SF]{subfigure}
%\usepackage{subfigure} 
\usepackage{times}
\usepackage{url}
\usepackage{verbatim} 
\usepackage{wrapfig}
\usepackage{xspace}
\usepackage{color}

\definecolor{grey}{RGB}{200,200,200}
\newcommand{\hilite}[1]{\colorbox{yellow}{#1}}
\newcommand{\hiliting}[1]{\colorbox{grey}{#1}}
\long\def\todo#1{\hilite{{\bf TODO:} {\em #1}}}

\setlength{\oddsidemargin}{0in}
\setlength{\textwidth}{6.5in}


% new commands: from file -- to permit re-use
%\input{newcommands.tex}

%Hyphenation problems
\hyphenation{name-space}
\hyphenation{schema-location}
\hyphenation{xml-lint}
\hyphenation{time-stamp}
\hyphenation{time-stamps}
\hyphenation{time-stamped}


\urlstyle{rm}
\graphicspath{{./figures/}}

\definecolor{gray}{rgb}{0.5,0.5,0.5}
%\newcommand{\rev}[1]{\textcolor{blue}{#1}}
%\newcommand{\quo}[1]{\textcolor{gray}{#1}}
%\newcommand{\rev}[1]{\vspace{3mm} \noindent {\em {\color{blue}{#1}}}}


\newenvironment{myindentpar}[1]%
{\begin{list}{}
         {\vspace{10pt}
					\setlength{\leftmargin}{#1}}
          \item[]
}
{\end{list}}
\newcommand{\rev}[1]{\begin{myindentpar}{.25in} {\em {\color{blue}{#1}}}\end{myindentpar}}

\newenvironment{myindentparQ}[1]%
{\begin{list}{}
         {\setlength{\leftmargin}{#1}}
          \item[]
}
{\end{list}}
\newcommand{\quo}[1]{\begin{myindentparQ}{.25in} {{\color{gray}{#1}}}\end{myindentparQ}}

\begin{document}

\title{Author's Reply}
\author{}
\maketitle

\section*{Overview}\label{sec:overview}
I appreciate your review and address your remaining question below. 
Note that my response to your question is 
incorporated into the revision and marked in {\color{blue}blue}.

\newcounter{RP}

\clearpage
\section*{Reviewer}\label{sec:rev1}

% 1
\rev{
$<<$ Reviewer's comments to the author(s) $>>$

The manuscript is well revised.

However, I have a question how to estimate each daemon's influence as time.
The estimation is important for accuracy of your proposed method.

For acceptance, please clarify the estimation method and its validity.

}

If I understand your question right, 
we actually *measure* each daemon's process time (PT) for each run of a program-under-test, 
rather than estimate its influence as time. 
PT is computed based on the measures of user and system time received from the kernel via 
the Netlink socket. 

To see whether the ``influence" (in your term) was significant on the timing, 
using the PT data we first identify infrequent, long-running daemons and then compute the *cutoff* of each so-identified daemon. 
As exhibited in Table~1, we obtain the detailed cutoff information for our identified infrequent, long-running daemon. 
The computed cutoffs are used to determine whether to discard a sample involving 
a daemon whose PT is above the respective cutoff, meaning that such a daemon's significant influence was there. 
The identification and cutoff computation are described from Steps~2 through~9, marked in blue, in Figure~3. 
For more details, please see the prose in Section~2 marked in blue. 

The validity of our protocol using the cutoff was assessed on real-world programs and the SPEC CPU benchmark. 
As shown in Figure~5 and Table~3, we could obtain better measurement quality for those real workloads, compared to 
the existing timing scheme using elapsed time.  

Note that to address your question, 
I slightly modified the Contribution paragraph for better clarification, and provided 
a single-line comment for better understanding about the cutoff computation. 

If I didn't catch your question correctly, please kindly let me know so that I will re-address it. 

\end{document}
