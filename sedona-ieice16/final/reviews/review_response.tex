\documentclass[10pt,letterpaper]{article}

\usepackage{graphicx}
\usepackage{longtable}
\usepackage{algorithmic}
\usepackage{alltt}
\usepackage{array}
\usepackage[cmex10]{amsmath}
\usepackage{amssymb}
%\usepackage[caption=false]{caption}
\usepackage[dvipsnames]{color}
%\usepackage{colortbl}
%\usepackage{enumitem}
%\usepackage{eqparbox}
%\usepackage{fixltx2e}
%\usepackage{float}
%\usepackage{floatflt}
%\usepackage{mdwmath}
%\usepackage{mdwtab}
%\usepackage{multirow}
\usepackage{stfloats}
%\usepackage[font=footnotesize]{subfig}
\usepackage[caption=false,font=normalsize,labelfont=sf,textfont=sf]{subfig}
%\usepackage[tight,normalsize,sf,SF]{subfigure}
%\usepackage{subfigure} 
\usepackage{times}
\usepackage{url}
\usepackage{verbatim} 
\usepackage{wrapfig}
\usepackage{xspace}
\usepackage{color}

\definecolor{grey}{RGB}{200,200,200}
\newcommand{\hilite}[1]{\colorbox{yellow}{#1}}
\newcommand{\hiliting}[1]{\colorbox{grey}{#1}}
\long\def\todo#1{\hilite{{\bf TODO:} {\em #1}}}

\setlength{\oddsidemargin}{0in}
\setlength{\textwidth}{6.5in}


% new commands: from file -- to permit re-use
%\newcommand{\ie}{\textit{i.e.},\xspace}
\newcommand{\eg}{\textit{e.g.},\xspace}
\newcommand{\cf}{\textit{cf.},\xspace}
\newcommand{\etc}{\textit{etc}.\@\xspace}
\newcommand{\vs}{\textit{vs}}
%\newcommand{\tVal}{{\sc $\tau$Validator}}
\newcommand{\tVal}{{\sc $\tau$XMLLint}}
\newcommand{\tX}{$\tau$XSchema}
\def\tB{\mbox{$\tau$Bench}}
\newcommand{\TODO}[1]{[TODO: #1]} 
\newcommand{\DONE}[1]{[DONE: #1]} 
\newcommand{\note}[1]{\colorbox{yellow}{\emph{#1}}}
\newcommand{\stevecomment}[1]{}
\newcommand{\tab}{\hspace{5mm}}



%%%%%%%%%%%%%%%%%%%%%%%%%%%%%%%%%%%%%%%%%%%%%%%%%%
%%% Commands taken from Steve's Thesis %%%%%%%%%%%%%%%%%%%%%%%%%%
\newcommand{\tool}[1]{{\sc #1}}
%\newcommand{\ex}[1]{ \hbox{\tt\small<#1>}} 
%\newcommand{\att}[1]{ \hbox{\tt\small#1}} 
\newcommand{\ele}[1]{\hbox{\tt <#1>}} 
\newcommand{\att}[1]{\hbox{\tt #1}} 
\newcommand{\attval}[1]{{\tt "#1"}} 
\newcommand{\tXml}{{\sc $\tau$XMLLint}}
\newcommand{\xmllint}{{\sc XMLLint}}
\newcommand{\squash}{{\sc Squash}}
\newcommand{\unsquash}{{\sc UnSquash}}
\newcommand{\resquash}{{\sc ReSquash}}
\newcommand{\validator}{{\em Temporal Constraint Validator Module}}
\newcommand{\schemamapper}{{\sc SchemaMapper}}
\newcommand{\putFile}[1]{\lstset{caption={\tt #1},label=listing:#1} \lstinputlisting{examples/#1}}
\newcommand{\putFileNL}[1]{\lstset{caption={\tt #1},label=listing:#1,numbers=none} \lstinputlisting{examples/#1} \lstset{label=,numbers=left}}
\newcommand{\putFileNLCompany}[1]{\lstset{language=CompanyXML} \lstset{caption={\tt #1},label=listing:#1,numbers=none} \lstinputlisting{examples/#1} \lstset{label=,numbers=left}\lstset{language=SteveXML} }
\newcommand{\putFileRep}[1]{\lstset{caption={\tt #1},label=listing:#1} \lstinputlisting{examples/exampleRep/#1}}
\newcommand{\putFileRepCapNL}[2]{\lstset{caption=#2,label=listing:#1,numbers=none} \lstinputlisting{examples/#1} \lstset{label=,numbers=left}}
\newcommand{\putFileRepCap}[2]{\lstset{caption=#2,label=listing:#1} \lstinputlisting{examples/exampleRep/#1} \lstset{label=}}

\newcommand{\needcite}{\colorbox{yellow}{\textbf{[?]}}}

% Counters for enumerate - makes bold letters (eg, (a)) and bold first few words.
\newcommand{\litem}[1]{\item{\bfseries #1}}
\newcounter{enumi_saved}
\newcommand{\startlist}{\begin{enumerate}[label=\textbf{(\alph{*})}]  \setcounter{enumi}{\value{enumi_saved}}}
\newcommand{\listdone}{\setcounter{enumi_saved}{\value{enumi}}\end{enumerate}}

% Counters for enumerate - makes bold letters (eg, (a)) and bold first few words.
\newcounter{denumi_saved}
\newcommand{\startlistD}{\begin{enumerate}[label=\textbf{(\arabic{*})}]\setcounter{enumi}{\value{denumi_saved}}}
\newcommand{\listdoneD}{\setcounter{denumi_saved}{\value{enumi}}\end{enumerate}}

% Smaller lists (bullets)
\newcommand{\squishlist}{
   \begin{list}{$\bullet$}
    { \setlength{\itemsep}{0pt}      \setlength{\parsep}{1pt}
      \setlength{\topsep}{1pt}       \setlength{\partopsep}{0pt}
      \setlength{\leftmargin}{2.8em} \setlength{\labelwidth}{1em}
      \setlength{\labelsep}{0.5em} } }
      
% Smaller lists (dash)
\newcommand{\squishdash}{
   \begin{list}{-}
    { \setlength{\itemsep}{0pt}      \setlength{\parsep}{1pt}
      \setlength{\topsep}{1pt}       \setlength{\partopsep}{0pt}
      \setlength{\leftmargin}{2.8em} \setlength{\labelwidth}{1em}
      \setlength{\labelsep}{0.5em} } }
\newcommand{\squishend}{
    \end{list}  }

% Smaller lists (numbers)
\newcommand{\numsquishlist}{
   \begin{list}{$\bullet$}
    { \setlength{\itemsep}{0pt}      \setlength{\parsep}{1pt}
      \setlength{\topsep}{1pt}       \setlength{\partopsep}{0pt}
      \setlength{\leftmargin}{2.8em} \setlength{\labelwidth}{1em}
      \setlength{\labelsep}{0.5em} } }

\newcommand{\numsquishend}{
    \end{list}  }


%Hyphenation problems
\hyphenation{name-space}
\hyphenation{schema-location}
\hyphenation{xml-lint}
\hyphenation{time-stamp}
\hyphenation{time-stamps}


%Hyphenation problems
\hyphenation{name-space}
\hyphenation{schema-location}
\hyphenation{xml-lint}
\hyphenation{time-stamp}
\hyphenation{time-stamps}
\hyphenation{time-stamped}


\urlstyle{rm}
\graphicspath{{./figures/}}

\definecolor{gray}{rgb}{0.5,0.5,0.5}
%\newcommand{\rev}[1]{\textcolor{blue}{#1}}
%\newcommand{\quo}[1]{\textcolor{gray}{#1}}
%\newcommand{\rev}[1]{\vspace{3mm} \noindent {\em {\color{blue}{#1}}}}


\newenvironment{myindentpar}[1]%
{\begin{list}{}
         {\vspace{10pt}
					\setlength{\leftmargin}{#1}}
          \item[]
}
{\end{list}}
\newcommand{\rev}[1]{\begin{myindentpar}{.25in} {\em {\color{blue}{#1}}}\end{myindentpar}}

\newenvironment{myindentparQ}[1]%
{\begin{list}{}
         {\setlength{\leftmargin}{#1}}
          \item[]
}
{\end{list}}
\newcommand{\quo}[1]{\begin{myindentparQ}{.25in} {{\color{gray}{#1}}}\end{myindentparQ}}

\begin{document}

\title{Author's Reply}
\author{}
\maketitle

\section*{Overview}\label{sec:overview}
I appreciate your review and address the last question below. 
Note that I mark all the changes made to this revision to respond to your concern are marked in {\color{blue}blue}. 
I hope that these changes helped you better read the flow of the paper. 


\newcounter{RP}

\clearpage
\section*{Reviewer}\label{sec:rev1}

% 1
\rev{
$<<$ Reviewer's comments to the author(s) $>>$

The manuscript is well revised.

However, I have a question how to estimate each daemon's influence as time.
The estimation is important for accuracy of your proposed method.

For acceptance, please clarify the estimation method and its validity.

}

If I understood your question right, 
each daemon's influence is translated to its respective cutoff, which I proposed and derived in this paper. 
For the detailed cutoffs, please refer to Tab.~1 and for computation process (estimation method in your term), the second paragraph in the right column on page~3.
As mentioned in the paper, the cutoffs are used to drop any execution (sample) involving daemon executions 
over their respective cutoffs. This drop indicates that such an execution was *significantly influenced* by infrequent, long-running daemons and thus we cannot include 
that execution in the computation of the execution time of a given program.

To assess the validity of my protocol using the cutoffs, I used the SPEC CPU2006 benchmark suite and observed that the protocol 
outperformed the extant method using elapsed time, as exhibited in Tab.~3 on page~4 (as already indicated by the previous revision). 
Note also that I took out Fig.~1 and instead moved Fig.~5 in the former revision (now Fig.~1) upfront (on page~1) in this revision, 
to early draw the attention of the protocol's effectiveness on two real programs (insertion sort and matrix multiplication) with increasing workloads.

Last but not least, there seems a confusion from the used program-under-test (PUT) (e.g., PUT128 and PUT16384) 
for not only defining but also evaluating the protocol. (In fact, the PUT is also an arbitrary \hbox{compute-bound} program, 
so there's no problem with applying the protocol to that. But I acknowledge that the dual use of the PUT potentially confuses readers including you.)
To unravel the confusion, I have redesigned the protocol, divided into the three major steps, as shown in Fig.~2 on page~2. 
In the new design, I talk about the PUT, only in the context of protocol definition (Step~B), but do not use it for evaluation. 
As described above, the evaluation was performed on those two real-world programs and the SPEC CPU benchmarks.

Many thanks for your great comments.

\end{document}
