% Cover letter using letter.sty
\documentclass{letter} % Uses 10pt
%Use \documentstyle[newcent]{letter} for New Century Schoolbook postscript font
% the following commands control the margins:
\topmargin=-1in    % Make letterhead start about 1 inch from top of page 
\textheight=8in  % text height can be bigger for a longer letter
\oddsidemargin=0pt % leftmargin is 1 inch
\textwidth=6.5in   % textwidth of 6.5in leaves 1 inch for right margin

\begin{document}

\signature{Young-Kyoon Suh}           % name for signature 
\longindentation=0pt                       % needed to get closing flush left
\let\raggedleft\raggedright                % needed to get date flush left
 
\begin{letter}{
%Editors-in-Chief of ACM Transactions on Computer Systems (TOCS): \\
%\begin{tabular}{ll}
%\vspace{.1in}
%
%Prof. Todd C. Mowry\\
%Computer Science Department\\
%Carnegie Mellon University\\
%5000 Forbes Avenue\\
%Pittsburgh, PA 15213 \\
%USA\\ 
}

\begin{flushleft}
{\large\bf Young-Kyoon Suh}
\end{flushleft}
\medskip\hrule height 1pt
%\begin{flushright}
%\hfill Department of Computer Science, The University of Arizona, Tucson, AZ 85721 USA\\
%\hfill Tel: +1-520-370-1865
%(Please understand we cannot reveal our affiliation, as the manuscript should be double-blind.)
%\end{flushright} 
\begin{flushright}
\hfill Supercomputing R\&D Center, 
\end{flushright}
\vspace{-.15in}
\begin{flushright}
\hfill Korea Institute of Science and Technology Information (KISTI), 
\end{flushright}
\vspace{-.15in}
\begin{flushright}
\hfill Republic of Korea, 34141, 
\end{flushright}
\vspace{-.15in}
\begin{flushright}
\hfill Tel: +82-42-869-0725
\end{flushright}
\vspace{-.15in}
\begin{flushright}
\hfill Email: yksuh@kisti.re.kr
\end{flushright} 
%\vfill % forces letterhead to top of page

%\vspace{-0.5in}
 
\opening{Dear Editor-in-Chief:} 
 
\noindent This letter to be submitted to IEICE Transactions on Information and Systems 
presents the following contributions. No version of this letter was submitted earlier. 

\begin{itemize}

\item We provide empirical evidence that measuring program time 
can be seriously affected by extant system daemons.

\item We propose a novel timing protocol, called {\em SEDONA} (Selective Elimination through Detection of infrequent, lOng-ruNning dAemons), 
that identifies infrequent, long-running daemons that impact the timing results for that program. 

\item We evaluate the performance of the protocol with rigorous experiments, starting from a simple program in pure-computation mode 
to a popular industrial benchmark suite.

\item Experimental results show a strong support for the effectiveness of our SEDONA timing protocol. 
\end{itemize} 
 
\closing{Sincerely yours,} 

\end{letter}
 

\end{document}