\documentclass[10pt,letterpaper]{article}

\usepackage{graphicx}
\usepackage{longtable}
\usepackage{algorithmic}
\usepackage{alltt}
\usepackage{array}
\usepackage[cmex10]{amsmath}
\usepackage{amssymb}
%\usepackage[caption=false]{caption}
\usepackage[dvipsnames]{color}
%\usepackage{colortbl}
%\usepackage{enumitem}
%\usepackage{eqparbox}
%\usepackage{fixltx2e}
%\usepackage{float}
%\usepackage{floatflt}
%\usepackage{mdwmath}
%\usepackage{mdwtab}
%\usepackage{multirow}
\usepackage{stfloats}
%\usepackage[font=footnotesize]{subfig}
\usepackage[caption=false,font=normalsize,labelfont=sf,textfont=sf]{subfig}
%\usepackage[tight,normalsize,sf,SF]{subfigure}
%\usepackage{subfigure} 
\usepackage{times}
\usepackage{url}
\usepackage{verbatim} 
\usepackage{wrapfig}
\usepackage{xspace}
\usepackage{color}

\definecolor{grey}{RGB}{200,200,200}
\newcommand{\hilite}[1]{\colorbox{yellow}{#1}}
\newcommand{\hiliting}[1]{\colorbox{grey}{#1}}
\long\def\todo#1{\hilite{{\bf TODO:} {\em #1}}}

\setlength{\oddsidemargin}{0in}
\setlength{\textwidth}{6.5in}


% new commands: from file -- to permit re-use
%\newcommand{\ie}{\textit{i.e.},\xspace}
\newcommand{\eg}{\textit{e.g.},\xspace}
\newcommand{\cf}{\textit{cf.},\xspace}
\newcommand{\etc}{\textit{etc}.\@\xspace}
\newcommand{\vs}{\textit{vs}}
%\newcommand{\tVal}{{\sc $\tau$Validator}}
\newcommand{\tVal}{{\sc $\tau$XMLLint}}
\newcommand{\tX}{$\tau$XSchema}
\def\tB{\mbox{$\tau$Bench}}
\newcommand{\TODO}[1]{[TODO: #1]} 
\newcommand{\DONE}[1]{[DONE: #1]} 
\newcommand{\note}[1]{\colorbox{yellow}{\emph{#1}}}
\newcommand{\stevecomment}[1]{}
\newcommand{\tab}{\hspace{5mm}}



%%%%%%%%%%%%%%%%%%%%%%%%%%%%%%%%%%%%%%%%%%%%%%%%%%
%%% Commands taken from Steve's Thesis %%%%%%%%%%%%%%%%%%%%%%%%%%
\newcommand{\tool}[1]{{\sc #1}}
%\newcommand{\ex}[1]{ \hbox{\tt\small<#1>}} 
%\newcommand{\att}[1]{ \hbox{\tt\small#1}} 
\newcommand{\ele}[1]{\hbox{\tt <#1>}} 
\newcommand{\att}[1]{\hbox{\tt #1}} 
\newcommand{\attval}[1]{{\tt "#1"}} 
\newcommand{\tXml}{{\sc $\tau$XMLLint}}
\newcommand{\xmllint}{{\sc XMLLint}}
\newcommand{\squash}{{\sc Squash}}
\newcommand{\unsquash}{{\sc UnSquash}}
\newcommand{\resquash}{{\sc ReSquash}}
\newcommand{\validator}{{\em Temporal Constraint Validator Module}}
\newcommand{\schemamapper}{{\sc SchemaMapper}}
\newcommand{\putFile}[1]{\lstset{caption={\tt #1},label=listing:#1} \lstinputlisting{examples/#1}}
\newcommand{\putFileNL}[1]{\lstset{caption={\tt #1},label=listing:#1,numbers=none} \lstinputlisting{examples/#1} \lstset{label=,numbers=left}}
\newcommand{\putFileNLCompany}[1]{\lstset{language=CompanyXML} \lstset{caption={\tt #1},label=listing:#1,numbers=none} \lstinputlisting{examples/#1} \lstset{label=,numbers=left}\lstset{language=SteveXML} }
\newcommand{\putFileRep}[1]{\lstset{caption={\tt #1},label=listing:#1} \lstinputlisting{examples/exampleRep/#1}}
\newcommand{\putFileRepCapNL}[2]{\lstset{caption=#2,label=listing:#1,numbers=none} \lstinputlisting{examples/#1} \lstset{label=,numbers=left}}
\newcommand{\putFileRepCap}[2]{\lstset{caption=#2,label=listing:#1} \lstinputlisting{examples/exampleRep/#1} \lstset{label=}}

\newcommand{\needcite}{\colorbox{yellow}{\textbf{[?]}}}

% Counters for enumerate - makes bold letters (eg, (a)) and bold first few words.
\newcommand{\litem}[1]{\item{\bfseries #1}}
\newcounter{enumi_saved}
\newcommand{\startlist}{\begin{enumerate}[label=\textbf{(\alph{*})}]  \setcounter{enumi}{\value{enumi_saved}}}
\newcommand{\listdone}{\setcounter{enumi_saved}{\value{enumi}}\end{enumerate}}

% Counters for enumerate - makes bold letters (eg, (a)) and bold first few words.
\newcounter{denumi_saved}
\newcommand{\startlistD}{\begin{enumerate}[label=\textbf{(\arabic{*})}]\setcounter{enumi}{\value{denumi_saved}}}
\newcommand{\listdoneD}{\setcounter{denumi_saved}{\value{enumi}}\end{enumerate}}

% Smaller lists (bullets)
\newcommand{\squishlist}{
   \begin{list}{$\bullet$}
    { \setlength{\itemsep}{0pt}      \setlength{\parsep}{1pt}
      \setlength{\topsep}{1pt}       \setlength{\partopsep}{0pt}
      \setlength{\leftmargin}{2.8em} \setlength{\labelwidth}{1em}
      \setlength{\labelsep}{0.5em} } }
      
% Smaller lists (dash)
\newcommand{\squishdash}{
   \begin{list}{-}
    { \setlength{\itemsep}{0pt}      \setlength{\parsep}{1pt}
      \setlength{\topsep}{1pt}       \setlength{\partopsep}{0pt}
      \setlength{\leftmargin}{2.8em} \setlength{\labelwidth}{1em}
      \setlength{\labelsep}{0.5em} } }
\newcommand{\squishend}{
    \end{list}  }

% Smaller lists (numbers)
\newcommand{\numsquishlist}{
   \begin{list}{$\bullet$}
    { \setlength{\itemsep}{0pt}      \setlength{\parsep}{1pt}
      \setlength{\topsep}{1pt}       \setlength{\partopsep}{0pt}
      \setlength{\leftmargin}{2.8em} \setlength{\labelwidth}{1em}
      \setlength{\labelsep}{0.5em} } }

\newcommand{\numsquishend}{
    \end{list}  }


%Hyphenation problems
\hyphenation{name-space}
\hyphenation{schema-location}
\hyphenation{xml-lint}
\hyphenation{time-stamp}
\hyphenation{time-stamps}


%Hyphenation problems
\hyphenation{name-space}
\hyphenation{schema-location}
\hyphenation{xml-lint}
\hyphenation{time-stamp}
\hyphenation{time-stamps}
\hyphenation{time-stamped}


\urlstyle{rm}
\graphicspath{{./figures/}}

\definecolor{gray}{rgb}{0.5,0.5,0.5}
%\newcommand{\rev}[1]{\textcolor{blue}{#1}}
%\newcommand{\quo}[1]{\textcolor{gray}{#1}}
%\newcommand{\rev}[1]{\vspace{3mm} \noindent {\em {\color{blue}{#1}}}}


\newenvironment{myindentpar}[1]%
{\begin{list}{}
         {\vspace{10pt}
					\setlength{\leftmargin}{#1}}
          \item[]
}
{\end{list}}
\newcommand{\rev}[1]{\begin{myindentpar}{.25in} {\em {\color{blue}{#1}}}\end{myindentpar}}

\newenvironment{myindentparQ}[1]%
{\begin{list}{}
         {\setlength{\leftmargin}{#1}}
          \item[]
}
{\end{list}}
\newcommand{\quo}[1]{\begin{myindentparQ}{.25in} {{\color{gray}{#1}}}\end{myindentparQ}}

\begin{document}

\title{Responses to Editor-in-Chief \& Reviewer Comments}
\author{}
\maketitle

\section*{Overview}\label{sec:overview}
We appreciate these quite thorough reviews. We appreciate the feedback and suggestions.

\newcounter{RP}

\clearpage
\section*{Editor-in-Chief}\label{sec:ed1}

Thank you for these detailed comments, which we respond to individually below.

\rev{
Comments from Editor:

15-Sep-2017

Dear Dr. Young-Kyoon Suh:

Manuscript ID TIIS-IS-2017-Jul-0844 entitled ``MLPPI Wizard: An Automated Multi-level Partitioning Tool on Analytical Workloads" which you submitted to the KSII Transactions on Internet and Information Systems, has been reviewed.  The comments of the reviewer(s) are included at the bottom of this letter.

I am happy to inform you that the reviewer(s) have recommended publication, but also suggest some revisions to your manuscript.  Therefore, I invite you to respond to the reviewer(s)' comments and revise your manuscript.   Please go through the reviewers' comments carefully and then prepare for the revised paper and authors' response. Your revised paper will not be guaranteed to be accepted for publication in the TIIS journal. The editor and reviewers will again review the revised paper and authors’ response.

}

In this response letter we address each of the reviewers' comments. 
We also incorporate into the revised paper our response to each comment. 

\rev{
To upload your revised manuscript, log into {\url{https://mc.manuscriptcentral.com/tiisjournal}} and enter your Author Center, where you will find your manuscript title listed under ``Manuscripts with Decisions."  Under ``Actions," click on ``Create a Revision."  Your manuscript number has been appended to denote a revision.
}

We used this procedure for uploading the revision.

\rev{
You may also click the below link to start the revision process (or continue the process if you have already started your revision) for your manuscript. If you use the below link, you will not be required to login to ScholarOne Manuscripts.

*** PLEASE NOTE: This is a two-step process. After clicking on the link, you will be directed to a webpage to confirm. ***

{\url{https://mc.manuscriptcentral.com/tiisjournal?URL_MASK=963223d1218947b5b5750544c56a74b6}}
}

Thank you for this additional information.

\pagebreak

\rev{
1. Revise your manuscript "by using the MS Word."

2. Please highlight the changes to your manuscript within the document by using bold and colored text "in BLUE" or by using the track changes mode in MS Word.

3. In addition to the revised manuscript, the authors are required to write the authors’ response answering the reviewers’ comments. In order to expedite the processing of the revised manuscript, please be as specific as possible in your response to the reviewer(s)' comments.

4. The authors’ response should be included in "the first page of the revised manuscript."

Once the revised manuscript is prepared, you can upload it and submit it through your Author Center.
}

Yes, we used the MS word program for revising our article. As requested, anything new or revised compared to the original submission is colored in {\color{blue}blue} and marked in \textbf{bold} face in the revised. In addition, corrections made are colored in {\color{red}red} for distinction. 
As required, we provide this written response letter to answer each reviewer's comments.
Finally, we include the letter in the first page of the revised article. 

\rev{
[IMPORTANT]:  Your original files are available to you when you upload your revised manuscript.  Please delete any redundant files before completing the submission.

Because we are trying to facilitate timely publication of manuscripts submitted to the KSII Transactions on Internet and Information Systems, your revised manuscript should be uploaded "by 15-Oct-2017."  If it is not possible for you to submit your revision in the deadline, we may have to consider your paper as a new submission.

Once again, thank you for submitting your manuscript to the KSII Transactions on Internet and Information Systems and I look forward to receiving your revision.

Sincerely,

Dr. Mohammad Shojafar

Editor,

mohammad.shojafar@uniroma1.it

KSII Transactions on Internet and Information Systems

WWW.ITIIS.ORG
}

We deleted any redundant files during the uploading process. Today ({\bf\underline{October 8th, 2017 in KST}}), the revised manuscript has been uploaded for further review, 
and thus, there is no issue with the specified deadline of October 15, 2017.

\clearpage
\section*{Reviewer 1}\label{sec:rev1}

% 1
\rev{
[Reviewer(s)' Comments to Author]:
Reviewer: 1

Comments to the Author
Generally Fig1. is not a figure scale and can be considered as a structure rather aha a Fig. 1. technically, the primary figures are the general scale and structure of the method which tends to grasp the authors' attention and must be clarified. 
}

\rev{
The presented Wizard Architecture in Fig. 2 needs to be expanded. 
}

\rev{
The table in Page3 does not have any caption to introduces besides, its location is badly organized.
}


\rev{
The main contribution of the paper is so limited and needs to be extended in Section 1. 
} 

\rev{ 
In the experiment environmental setting, the deployment in realistic datasets are missing and must be indicated for the reproduction of the future readers.  According to the bar-shaped results, the WIZARD does not provide optimal results compared to the state-of the arts. They must clearly be expressed how this cases can be used in the real-time scenarios. 
}
 

\rev{
Also, Authors should give a more clear definition of the "application type" used in the paper. For example, what information of applications are used and how to obtain this information. 
}

\rev{
Last but not least backs to the background which is limited and some hot-related works are missing that are listed in i)"Using imperialist competition algorithm for independent task scheduling in grid computing" and ii) "A hybrid metaheuristic algorithm for Job scheduling on computational grids" that addressed the analytical workload exploitation in the real case studies that must be added in the background to make your work holistic for the future readers.
}

\clearpage
\section*{Reviewer 2}\label{sec:rev2}

% 2
\rev{
Reviewing: 2

Comments to the Author
The paper could be published after addressing following comments:
1. Abstract could be better write, in this reviewer point of view, the current abstract is not comprehensive.
}
 
\rev{
2. The manuscript could be better organize as well. For instance, the related works section could be placed after introduction section no before conclusion.
}

\rev{
3. Providing a comparison table could be useful for readers.
}

\rev{
4. Conclusion is too long.
}

\rev{
5. Introduction section need more references.
}

\end{document}
